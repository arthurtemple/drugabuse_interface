\documentclass[a4paper,12pt, twocolumn]{article}
\usepackage[T1]{fontenc}
\usepackage[top=3cm, bottom=2cm, left=2cm, right=2cm]{geometry}
\usepackage{fancyhdr}
\pagestyle{fancyplain}

\fancyhead{} % clear all header fields
% \renewcommand{\headrulewidth}{0pt} % no line in header area
\fancyfoot{} % clear all footer fields
% \fancyhead[LE,RO]{\thepage}           % page number in "outer" position of footer line
\fancyhead[RE,LO]{\emph{UI Construction - Group assignment report - Olivier Cl\'{e}ro, Maciek \'{S}lifirski, Arthur Templ\'{e}} } % other info in "inner" position of footer line

\title{T-110.5241 - Report for the group assignment}

\begin{document}

\maketitle

\section*{Introduction}

The group is formed by Maciek \'{S}lifirski, Olivier Cl\'{e}ro and Arthur Templ\'{e}.

We chose topic number 6: \textbf{Anonymous substance abuse counseling e-service}.

\section*{The service - Anonymous substance abuse counseling e-service}

The service is meant to provide information to people in difficulty regarding drug abuse.
The aim is to give:
\begin{enumerate}
 \item proper answers to people's drug problems
 \item easy access to common services (like the website, emergency calls, ...)
\end{enumerate}

For this purpose, four user interfaces are provided:

\begin{itemize}
 \item Web
 \item Mobile
 \item Desktop
 \item Command Line
\end{itemize}

Furthermore, a placeholder backend server is also incldued, which provides PDF files according to users' requests. Basically, it searches in the request (which is a text string) for drug names, and crafts a proper PDF file accordingly to the identified drugs.

\section*{Web Interface}

\section*{Mobile Interface}

\section*{Desktop Interface}

\section*{Command Line Interface}

\subsection*{Overview}

The Command Line Interface (CLI) version of this project may be considered a reduced port of the Desktop Interface version for CLI, towards users not desiring to use any graphical environment. Functionalities are stripped down to essential. User is able to open a browser to the web interface, to an information website and to send a request to the server and get the informative PDF file he requested. Furthermore, the CLI allows opening this PDF file with default PDF reader.

\subsection*{Advantages and issues from the CLI}



\subsection*{Technical choices}



\subsection*{Outcomes}

\section*{Conclusion}

\begin{thebibliography}{9}

\bibitem{foobar2000}
  Bar, F. and Qux, B. 2000. \emph{how do you turn this on}.	% May actually use APA-style citing, as provided by Google Scholar for instance
  
\end{thebibliography}

\end{document}
